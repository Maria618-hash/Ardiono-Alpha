% Paper-ready Results/Evaluation section (IEEE style).
% Drop this into your main IEEEtran document and adjust figure numbering as needed.
%
% Recommended packages in main preamble:
% \usepackage{graphicx}
% \usepackage{booktabs}
% \usepackage{multirow}
%
% Figure paths are set in main.tex via \graphicspath{{figures/}}.

\section{Results and Evaluation}
\label{sec:results}

\subsection{Evidence Artifacts and Experimental Workflow}
Evaluation was conducted at two levels: (i) \emph{toolchain-level validation} confirming that the custom authorization instruction is supported end-to-end (compile, assemble, link, and produce hex images), and (ii) \emph{system-level validation} confirming that both execution and debug access are gated by authorization and that failures result in a fail-stop halt.
Evidence was collected from the host-side application logs and screenshots extracted from the development artifacts.

\subsection{Toolchain Validation}
Fig.~\ref{fig:compile} shows compilation of a test program that invokes the custom instruction via inline assembly, along with successful ELF-to-HEX generation reported by the tool.
This demonstrates that the custom instruction can be emitted from C code and preserved across the build pipeline without manual binary patching.

\begin{figure}[t]
  \centering
  \includegraphics[width=\linewidth]{fig_compile.png}
  \caption{Toolchain validation: successful compilation and ELF$\rightarrow$HEX generation with a test program invoking the custom instruction.}
  \label{fig:compile}
\end{figure}

\subsection{Firmware Inspection (HEX Viewer)}
To support operator verification and troubleshooting, the host application provides a hex viewer for generated images.
Fig.~\ref{fig:hex} shows an example view of the generated hex contents.

\begin{figure}[t]
  \centering
  \includegraphics[width=\linewidth]{fig_hex.png}
  \caption{Firmware inspection: hex viewer output for the compiled image.}
  \label{fig:hex}
\end{figure}

\subsection{Authorization and Debug Enablement (Key Match)}
The core functional requirement is that debug access is released only after successful authorization.
Fig.~\ref{fig:keymatch} shows the key exchange and matching evidence in the application log, and Fig.~\ref{fig:debugenabled} shows the resulting ``Debugger is enabled'' indication after a successful match.

\begin{figure}[t]
  \centering
  \includegraphics[width=\linewidth]{fig_keymatch.png}
  \caption{Authorization evidence: key exchange and match indication in the host application log.}
  \label{fig:keymatch}
\end{figure}

\begin{figure}[t]
  \centering
  \includegraphics[width=\linewidth]{fig_debugenabled.png}
  \caption{Debug gating: debugger enablement occurs only after successful key match.}
  \label{fig:debugenabled}
\end{figure}

\subsection{Connection and Programming Workflow}
Fig.~\ref{fig:connected} shows an example ``Connected Successfully'' state in the programming interface.
Fig.~\ref{fig:loadimage} shows a programming session log including the \texttt{load\_image} operation for a firmware ELF, demonstrating an end-to-end workflow from compilation to device programming.

\begin{figure}[t]
  \centering
  \includegraphics[width=\linewidth]{fig_connected.png}
  \caption{Connection status: host tool indicates successful device connection.}
  \label{fig:connected}
\end{figure}

\begin{figure}[t]
  \centering
  \includegraphics[width=\linewidth]{fig_loadimage.png}
  \caption{Programming workflow evidence: host log includes \texttt{load\_image} for the compiled firmware ELF.}
  \label{fig:loadimage}
\end{figure}

\subsection{Fail-Stop Enforcement (Halt-on-Failure)}
The security policy requires halting the processor on authorization failure while keeping execution and debug disabled.
Fig.~\ref{fig:halted} provides evidence consistent with a fail-stop halt state.

\begin{figure}[t]
  \centering
  \includegraphics[width=\linewidth]{fig_halted.png}
  \caption{Fail-stop behavior: processor indicates a halted state consistent with halt-on-failure enforcement.}
  \label{fig:halted}
\end{figure}

\subsection{Summary}
Table~\ref{tab:results-summary} summarizes observed behaviors supported by the collected evidence.

\begin{table}[t]
  \centering
  \caption{Functional validation summary from collected artifacts.}
  \label{tab:results-summary}
  \begin{tabular}{@{}ll@{}}
    \toprule
    \textbf{Requirement} & \textbf{Evidence} \\
    \midrule
    Custom instruction compiles and emits & Fig.~\ref{fig:compile} \\
    ELF$\rightarrow$HEX generation succeeds & Fig.~\ref{fig:compile} \\
    Hex inspection supported & Fig.~\ref{fig:hex} \\
    Key exchange and match visible & Fig.~\ref{fig:keymatch} \\
    Debug enabled only after match & Fig.~\ref{fig:debugenabled} \\
    Device connection workflow supported & Fig.~\ref{fig:connected} \\
    Programming log shows image load & Fig.~\ref{fig:loadimage} \\
    Halt-on-failure evidence & Fig.~\ref{fig:halted} \\
    \bottomrule
  \end{tabular}
\end{table}

% Prevent floats from drifting into the Discussion section.
\FloatBarrier

