% Paper-ready Future Work (revised to match this project scope).

\section{Future Work}
\label{sec:futurework}

\subsection{Device-Bound Key Derivation and Provisioning Workflow}
While the current design binds authorization to host- and device-associated identifiers (e.g., UUIDs), a complete deployment requires a well-defined provisioning workflow. Future work will formalize enrollment steps for extracting/confirming a device identifier, generating the device-bound verifier, and securely provisioning the reference value into protected non-volatile storage. This includes defining recovery procedures for legitimate re-enrollment (e.g., device replacement) while preventing unauthorized re-binding.

\subsection{Database-Backed Policy and Audit at Scale}
The host-side database can be extended from a single binding store to a policy service that manages multiple devices, multiple authorized hosts, and role-based access (e.g., operator vs. administrator). Future work will add (i) firmware metadata storage (firmware hash/version), (ii) per-device authorization history, and (iii) reproducible build metadata (toolchain version, configuration), enabling stronger provenance and easier forensic analysis.

\subsection{Multi-Step Authorization (Key Sequencing)}
The design diagrams include a staged authorization sequence (e.g., Key-1 $\rightarrow$ Key-2 $\rightarrow$ Key-3). Future work will implement and evaluate multi-step authorization in the Authenticator FSM, including strict ordering, per-step timeout constraints, and explicit failure-to-halt semantics. This can improve robustness against partial replay and makes the authorization protocol more resistant to trivial single-token reuse.

\subsection{Hardware and Micro-Architectural Evaluation}
The current results emphasize functional correctness and workflow evidence. Future work will quantify overhead and design trade-offs, including (i) authorization latency (boot-time impact), (ii) area/power cost of the compare path and control FSM, and (iii) any code-size impact introduced by toolchain integration and BootROM logic.

\subsection{Toolchain Hardening and Regression Testing}
To make the ISA extension sustainable, future work will add automated regression tests that validate encoding/decoding, compilation, disassembly, and correct placement of the authorization primitive in early boot. This includes a small test suite that checks that binaries contain the expected instruction sequences and that failure modes (halt-on-failure) are triggered under incorrect tokens.

